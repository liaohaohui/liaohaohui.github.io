\documentclass[a4paper,12pt]{article}
\usepackage[margin=2cm]{geometry}
\usepackage{url}

\usepackage{natbib} % agsm

\usepackage{amsmath}
\usepackage{listings}
\lstset{
  basicstyle=\ttfamily\small,
  numbers=left,
  stepnumber=1,
}
\usepackage{epsfig}


%%% The project title is clearly stated and reflects the contents of the report
\title{Real-World Data Analysis}
\author{Liew How Hui}
\date{\sf January 2016}

\begin{document}


\maketitle
%\tableofcontents

\section{Objective and Scope}

%%% The objectives are clear and achievable.
This project plans to understand how to analyse real-world data or
problems using statistical software.

%%% The scope of the project is clearly stated.
By perform a Google search on ``real-world data analysis'', a lot of
``real-world data'' related to clinical data, big data pop up.  This
project is not going to explore those data but is going to limit the
scope to a UTAR data obtain during consultation and data organise from
workload problem.

%%% The rationales behind the project are discussed and they are significant.
It is important in an information society to be able to process
data efficiently with the right software tools.  This project is going
to contribute to the understanding of real-world data and problems
that one will encounter as well as to master the right software tools
to process real-world data.


%%% Make sure "the grammar is correct"



\section{Literature Review}

%%% The literature review/technical review is relevant to the project scopes.

A Google search on ``real-world data analysis'' returned something
many links on health study and big data as well as links to Coursera's
``Data Science in Real Life by Johns Hopkins University'', ``Python
for Data Analysis'', ``Practical Data Science Cookbook''.  We are
interested in the later, so we will limit the search to ``books on
data analysis'', which returns a link on ``free books for learning
data mining and data analysis'':
\begin{enumerate}
\item Data Jujitsu: The Art of Turning Data into Product
  (O'Reilly)
\item Data Mining Algorithms in R (Wikibook)
\item Data Mining and Analysis: Fundamental Concepts and Algorithms
\item Introduction to Data Science (Jeffrey Stanton, using R,
  Creative Common License)
\item Mining of Massive Datasets (www.mdds.org)
\item School of Data Handbook
\end{enumerate}
Note that a book is free does not mean that we can copy and paste from
it without permission but we can download and read the electronic
copy.

There are also so-called ``must-read'' (popular) books based on the
recommendation of some newspaper comments or Amazon users:
\begin{enumerate}
\item Big Data: A Revolution That Will Transform How We Live, Work,
  and Think
\item The Signal and the Noise: Why So Many Predictions Fail --- but
  Some Don't
\item Predictive Analytics: The Power to Predict Who Will Click,
  Buy, Lie, or Die
\item Web Scraping with Python: Collecting Data from the Modern Web
\end{enumerate}




\section{Data Analysis Procedure}

%%% Research methods and/or development tools are discussed to achieve the goals of the project.

%%% The methodologies and techniques used for designing, verifying and proving the proposed models/ algorithms/ theorems/ hypotheses are relevant to the problems.

How could we achieve our objective.  We already know that Excel is a
basic tool in data analysis.  However, Excel is not powerful enough to
handle large data.  We need more powerful software.  An Internet
search gave a list of data analysis software below:
\begin{itemize}
\item ELKI (Java)
\item Weka (Java)
\item DataMelt (Java, previously SCaVis)
\item ROOT, PAW (Physics Analysis WOrkStation) from CERN, written in
  C++, Fortran
\item R
\item Encog Machine Learning Framework
\item SPSS (IBM)
\item SAS
\item Stata
\end{itemize}

UTAR does have the license for the popular commercial software SAS
which can solve statistical and data analysis problems.  However, the
installation of the software take up a lot of space and the knowledge
gained from using it is only good when working in a large
corporation.

A better choice is probably a mixed of Excel and open source software
for data analytics.  Python and R are two very popular data analysis
software.  Syntax-wise, Python is a better choice.



\section{Mathematical Formulas in \LaTeX{}}

Linear regression formula
Let $\mathbf{y}$ be the respondent variable and $\mathbf{x}$ be the
factor.  Then ...
$$
\mathbf{y} = (A^TA)^{-1} A^T\mathbf{x}
$$

The matrix
$$
\begin{bmatrix}
%  a_{11} & a_{12} & a_{13} \\
  a_{21} & a_{22} & a_{23} \\
  a_{31} & a_{32} & a_{33} \\
  b_{31} & b_{32} & b_{33} & b_{34}
\end{bmatrix} \quad
\begin{pmatrix}
  a_{11} & a_{12} & a_{13} \\
  a_{21} & a_{22} & a_{23} \\
  a_{31} & a_{32} & a_{33} \\
\end{pmatrix}
\begin{vmatrix}
  a_{11} & a_{12} & a_{13} \\
  a_{21} & a_{22} & a_{23} \\
  a_{31} & a_{32} & a_{33} \\
\end{vmatrix}
$$

Integration and Differentiation
$$
\begin{aligned}
  F'(x) = \frac{dF}{dx} &= f(x)\\
  \int_a^x f(x) dx  &= F(x) - F(a)
\end{aligned}
$$
or special function
$$
y =
\begin{cases}
  f_1(x) & a\le x < c\\
  f_2(x) & c\le x \le b
\end{cases}
$$


%%% Programming Code can be inserted using lstinputlisting
%\lstinputlisting[frame=tb]{simple.py}


\section{Project Planning}

%%% The time frame and budget allocated (if any) are reasonable for each phase.
%%% The subtasks and phases towards the end of the project are properly defined.

\begin{center}
  \begin{tabular}{|c|p{10cm}|c|}
    \hline
    Week & \hfil Plans & Complete?\\
    \hline
    2 &
    \begin{itemize}
    \item Check out ``United Nations Statistics Division'' at
      \url{http://unstats.un.org/unsd/}
    \end{itemize} &
    \\
    3 & Learn software for gathering and reading data &
    \\
    4 & Use software to collect and gather real-word data &
    \\
    5 & Typing the review of literature and data into the proposal &
    \\
    6 & Clean up and print out the proposal.  After discussion with
    supervisor, start to transfer the results into the interim report
    \LaTeX{} file. &
    \\
    7--9 & Learn how to use software to clean and format data &
    \\
    10--11 & Type the above research result into the interim report
    and make a plan for the coming Project II.  Discuss with
    supervisor to correct some mistakes in the report for the
    submission in Week 12. &
    \\
    \hline
  \end{tabular}
\end{center}




%http://www.ctan.org/tex-archive/macros/latex/contrib/harvard
%\renewcommand{\bibname}{\textbf{\textsc{References}}}
%\bibliographystyle{agsm}
%\bibliography{bibdatabase}


\end{document}

%%% Local Variables:
%%% mode: latex
%%% TeX-master: t
%%% End:
